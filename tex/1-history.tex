
\chapter{作曲に関する経緯}

\section{背景と作曲過程}

Brahmsが晩年の室内楽分野における傑作, クラリネット三重奏曲とクラリネット五重奏曲を作曲したのは1891年で,
これはマイニンゲン宮廷管弦楽団の主席クラリネット奏者Richard Mühlfeldの演奏に触発されたものであることは広く知られている.
その後Brahmsはピアノのための小品 (作品116から119) に重点を移しているが,
クラリネットとピアノのための作品というアイディアは持っていたことがFerdinand Schumannの回想で伝わっている\cite{library}.
このアイディアが実現するのは1894年, Brahmsはイシュル滞在中の7月から8月にかけて2曲のクラリネットソナタを平行して作曲している\cite{compos}\cite{henle}.
ただし第1番第1楽章, 第3楽章のスケッチがイシュルへ移動する前に残されている\cite{library}\cite{henle}.
このふたつのソナタは1891年の作品とは異なりB管クラリネットのために書かれている.

この時期のBrahmsの周辺は栄光と親しい友人との死別で彩られている.
1892年にはElisabeth von Herzogenbergが1月7日に亡くなっているし, 1893年2月26日に女性歌手だったHermine Spiesが急逝している.
1894年の2月6日にはTheodor Billroth, 2月12日にはHans von Bülowと立て続けに二人の親友が亡くなり, さらに4月13日にはPhilipp Spittaが亡くなっている.
一方でBrahmsの名声は頂点にあり, 1893年の60歳の誕生日に際して大規模な祝典が企画されたが,
Brahmsはこれを辞退してWidmannやHanslick夫妻とイタリア旅行へ出かけている\footnote{回想録集第3間p.140-145に記述あり.}.
また, 若い頃に渇望していた地位であるハンブルク市のフィルハーモニー協会音楽監督への就任依頼が1894年4月に届く.
これらの体験は「枯淡の境地」と呼ばれる後期ピアノ曲集に反映されていると考えられている.
それに対して, 作品120はそれを突き抜けたシンプルさ, 明るさが特徴的である.
どちらのソナタも, MozartやMendelssohnを思わせる明快さ, そして洗練された作曲技法を備えている.
主題展開に関しても, 以前のような変奏の技法を駆使するよりも,
比較的原型を留めたままの形でごくわずかな変更で絶妙なニュアンスを出す方向へと舵を切っている.
これらの傾向は1996年の4つの厳粛な歌 op.121でさらに推し進められることになる.


\section{初演}

Brahmsは8月26日付の手紙でMühlfeldをイシュルへ招待し, 30日の手紙でその理由を明かしている\cite{henle}.
「私はあなたのためにクラリネット協奏曲を作るほど無謀ではありません! すべてが上手くいけば, ピアノ伴奏による控えめなソナタが二つ出来上がりそうです.」
結局Mühlfeldとは9月にベルヒテスガーデンで合流し, 1894年9月23日にマイニンゲン公爵らの立ち合いのもと, BrahmsはMühlfeldとこの2曲のソナタを試演している\cite{compos}.
続いて11月にはフランクフルトでClara Schumann, Joseph Joachimの前でも演奏するなど, 私的な演奏を重ねてこの曲を改訂している\cite{henle}.
その際に, Clara宅での演奏にJoachimを招待する10月14日付の手紙においてJoachimにヴィオラ譜も準備できていると伝えているので,
その時点までにヴィオラ版も作成されていると考えられる\cite{henle}.
公開初演はウィーンのBösendorfersaalにて, Rosé四重奏団のコンサートと併せて, 第2番が1895年1月8日,
第1番が1月11日に行われた\footnote{そのリハーサルに関する記述が回想録集第2巻p.123にある.}\cite{library}\cite{henle}.
恒例の演奏旅行がその後に続き, 3月までにベルリン, ライプツィヒ, フランクフルト, メルゼブルク, マイニンゲンで演奏している\cite{compos}.

このふたつのソナタは概して好評で, 例えば1月27日のライプツィヒ公演についてMusikalisches Wochenblatt紙は好意的に伝えている.
また, Eduard Hanslickはウィーンでの初演を聴いて次のように評している\cite{henle}.
「変ホ長調のソナタの第1楽章はとても喜ばしいものだった. テーマは天国から舞い降りてきたものか, もしくは最も美しい若き日々から漂い出してきたもののようだ.
甘美な恋, そして愛の恍惚とした喜びに満たされている! 序なしにクラリネットが語るこのメロディー, その歌に酔わされるから,
私はこの楽章が最良に思えたし, ヘ短調のものよりも変ホ長調のソナタを評価したい」(Neue Freie Presse, 15 Jan 1895)


\section{出版}

2月26日にマイニンゲンでこのソナタを演奏した後, BrahmsはSimrockに楽譜を送り, 出版に向けた作業を開始している.
校正作業は3月中に完了し, 最終的にBrahmsは校正済みヴィオラ譜を4月1日付でSimrockに送付している\cite{henle}.
出版は1895年6月中旬で, 同時にBrahms自身によるヴィオラ譜も出版されている.
出版のための作業は4月上旬には終わっていたが, 実際に発行されるまでに時間を設けることでMühlfeldの独占的な演奏を確保しているのである\cite{henle}.
Brahmsは同年夏にさらにピアノパートにも手を加えたヴァイオリンとピアノのための編曲版も出版している\cite{imslp}.

主要な出版譜は以下の通り.
\begin{itemize}
	\item Simrock社, 1895年 (クラリネット版Plate 10408-09, ヴィオラ版Plate 104010-11, ヴァイオリン版Plate 104033-34)
	\item Breitkopf \& Härtel社, 1926-27年, Hans Gál編 (Plate J.B.~41-42, 新装版EB 6915-18)
	\item Universal Edition, Hans-Christian Müller編 [ウィーン原典版] (UT50015-16)
	% \item インターナショナル・ミュージック社, 1997年 (No.~418-419)
	\item Henle社, 1974年, Monica Steegmann編 (クラリネット版HN274, ヴィオラ版HN231)
	\item Henle社, 2013年, Johannes Behr \& Egon Voss編 (クラリネット版HN987, ヴィオラ版HN988)
\end{itemize}
また, 第1番に関してはLuciano Berioによる管弦楽伴奏版 (UE18868, 1986年), 第2番はRainer Schottstädtによる管弦楽伴奏版 (GS 20 015, 1951年) がある.

自筆譜はニューヨークのモルガン・ライブラリーに所蔵されている. またWilliam Kupferによる筆写譜はStaats- und Universitätsbibliothek Hamburgにある.
Brahms自身が所有していた彼による書き込み付きSimrock初版譜は, 第2番のみウィーン楽友協会が収蔵している.
