
\chapter{クラリネットソナタ第2番}

% \section{全体の構造}

\section{第1楽章: Allegro amabile}

典型的なソナタ形式の楽章だが, 第1番に比べ形式的明瞭さが増している. 変ホ長調, 4分の4拍子.

\musicbegin
	\def\nbinstruments{2}%   % パート数 2
	\setstaffs{1}{2}%        % 下から1番目は2段
	\setclef{1}{6000}%       % 下から1番目はへ音記号
	\generalsignature{-3}%    % 調号は正の値のときシャープの数
	\generalmeter{\meterfrac{2}{4}}%  % 拍子は8分の6拍子
	\setname{2}{Cl\hspace{8truemm}}%
	\setname{1}{Pf\hspace{8truemm}}%
	\startextract%
		%(1)
		\notes\ibu{0}{G}{+2}\zhl{E}\qb{0}{E}\zqb{0}{I}\pt{H}\ql{I}\cu{**}\zh{I}\hu{E}|%
			\cu{**}\qb{0}{N}\tbl{0}\qb{0}{e}\ibl{1}{e}{-3}\qb{1}{gbe}\tbl{1}\qb{1}{N}&%
			\isluru{2}{l}\qlp{l**}\cl{k}\ibl{2}{j}{-1}\qb{2}{mlg}\tbl{2}\tslur{2}{j}\qb{2}{j}\enotes
		\bar
		%(2)
		\notes|%
			&%
			\ibsluru{2}{h}\hu{h***}\tbsluru{2}{d}\qu{d*}\ds\cu{c}\enotes
		\bar
		%(3)
		\notes|%
			&%
			\qup{e**}\cu{b}\ql{i*}\ds\cu{d}\enotes
	\endextract % 頭にzをつけると最後に小節線を表示しない
\musicend{2-1-1}{第2番第1楽章冒頭. クラリネットはB管だがここでは実音表記. }


序をおかずクラリネットによる第1主題の提示から始まる (譜例\ref{2-1-1}).


\section{第2楽章}

\section{第3楽章}
