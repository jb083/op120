
\chapter{クラリネットソナタ第2番}

% \section{全体の構造}

\section{第1楽章: Allegro amabile}

\begin{table}[htbp]
	\centering
	\begin{tabular}{ccc|ccc|ccc|c}
		\multicolumn{3}{c|}{提示部E} & \multicolumn{3}{c|}{展開部D} &
			\multicolumn{3}{c|}{再現部R} & コーダC \\ \hline
		\multicolumn{3}{c|}{1--55} & \multicolumn{3}{c|}{56--102} &
			\multicolumn{3}{c|}{103--149} & 150--173 \\
		\ind{E}{1} & \ind{E}{2} & \ind{E}{c} & \ind{D}{1} & \ind{D}{2} & \ind{D}{3} &
			\ind{R}{1} & \ind{R}{2} & \ind{R}{c} & \\ \hline
		1- & 22- & 40- & 56- & 73- & 88- & 103- & 120- & 138- & \\
		Es & B & B & Es--g & G & B & Es & Es & Es & E--Es
	\end{tabular}
	\caption{第1楽章の構成}
	\label{structure of mov1}
\end{table}


典型的なソナタ形式の楽章だが, Brahmsのどのソナタ楽章と比較しても形式的明瞭さが顕著である. 変ホ長調, 4分の4拍子.

\musicbegin
	\def\nbinstruments{2}%   % パート数 2
	\setstaffs{1}{2}%        % 下から1番目は2段
	\setclef{1}{6000}%       % 下から1番目はへ音記号
	\generalsignature{-3}%    % 調号は正の値のときシャープの数
	\generalmeter{\meterC}%  % 拍子
	\setname{2}{Cl\hspace{8truemm}}%
	\setname{1}{Pf\hspace{8truemm}}%
	\startextract%
		%(1)
		\notes\ibu{0}{G}{+2}\zhl{E}\qb{0}{E}\zqb{0}{I}\pt{H}\ql{I}\cu{**}\zh{I}\hu{E}|%
			\cu{**}\qb{0}{N}\tbl{0}\qb{0}{e}\ibl{1}{e}{-3}\qb{1}{gbe}\tbl{1}\qb{1}{N}&%
			\isluru{2}{l}\qlp{l**}\cl{k}\ibl{2}{j}{-1}\qb{2}{mlg}\tbl{2}\tslur{2}{j}\qb{2}{j}\enotes
		\bar
		%(2)
		\notes|%
			&%
			\ibsluru{2}{h}\hu{h***}\tbsluru{2}{d}\qu{d*}\ds\cu{c}\enotes
		\bar
		%(3)
		\notes|%
			&%
			\qup{e**}\cu{b}\ql{i*}\ds\cu{d}\enotes
	\endextract % 頭にzをつけると最後に小節線を表示しない
\musicend{2-1-1}{第2番第1楽章冒頭. クラリネットはB管だがここでは実音表記. }


序をおかずクラリネットによる第1主題の提示から始まる (譜例\ref{2-1-1}).

\musicbegin
	\shosetu{22}
	\def\nbinstruments{1}%   % パート数 1
	\setstaffs{1}{1}%        % 下から1番目は2段
	\setclef{1}{0000}%       % 下から1番目はへ音記号
	\generalsignature{-3}%    % 調号は正の値のときシャープの数
	% \generalmeter{\meterC}%  % 拍子
	\startextract%
		\notes\islurd{0}{M}\qu{M*}\tslur{0}{f}\qu{f*}\ds\ibu{0}{f}{+1}\islurd{0}{f}\qb{0}{ff}\tbu{0}\tslur{0}{g}\qb{0}{g}\enotes
		\bar
		\notes\islurd{0}{g}\qu{g*}\tslur{0}{c}\qu{c*}\ds\ibu{0}{f}{0}\islurd{0}{c}\qb{0}{cf}\tbu{0}\tslur{0}{c}\qb{0}{c}\enotes
		\bar
		\notes\islurd{0}{d}\qup{d**}\tslur{0}{b}\cu{b}\islurd{0}{b}\qup{b**}\na{a}\tslur{0}{a}\cu{a}\enotes
		\bar
		\Notes\na{a}\islurd{0}{a}\qu{a}\qu{b}\tslur{0}{c}\qu{c}\qp\enotes
	\endextract % 頭にzをつけると最後に小節線を表示しない
\musicend{2-1-22}{第2番第1楽章第22小節から.}

\musicbegin
	\shosetu{40}
	\def\nbinstruments{1}%   % パート数 1
	\setstaffs{1}{1}%        % 下から1番目は2段
	\setclef{1}{0000}%       % 下から1番目はへ音記号
	\generalsignature{-3}%    % 調号は正の値のときシャープの数
	% \generalmeter{\meterC}%  % 拍子
	\startextract%
		\Notes\qp\isluru{0}{p}\ql{p}\na{o}\ql{o}\tslur{0}{l}\ql{l}\enotes
		\bar
		\notes\ibl{0}{m}{0}\na{o}\isluru{0}{o}\qb{0}{onk}\tbl{0}\tslur{0}{n}\qb{0}{n}\isluru{0}{m}\ql{m*}\tslur{0}{j}\ql{j*}\enotes
		\bar
		\notes\ds\ibl{0}{k}{-2}\isluru{0}{k}\qb{0}{kl}\tbl{0}\tslur{0}{i}\qb{0}{i}\enotes
		\nnotes\ibbl{0}{j}{-1}\isluru{0}{k}\qb{0}{kj}\fl{g}\qb{0}{g}\tbl{0}\tslur{0}{j}\qb{0}{j}%
			\ibbl{0}{h}{+2}\isluru{0}{f}\qb{0}{fik}\tbl{0}\tslur{0}{m}\qb{0}{m}\enotes
		\bar
		\Notes\ql{p}\enotes
	\zendextract % 頭にzをつけると最後に小節線を表示しない
\musicend{2-1-40}{第2番第1楽章第40小節から.}

\section{第2楽章}

\section{第3楽章}
