\documentclass[lualatex,a4paper,openany]{bxjsbook}
\usepackage{amsmath,amssymb,mathrsfs,braket}
% \usepackage{amsthm,amscd} %定理環境, 可換図式
% \usepackage{ascmac} %screen, itembox環境
\usepackage{graphicx,xcolor}
% \usepackage{fancyhdr,lastpage} %ヘッダー/フッター操作
\usepackage{multicol} % 2段組
\usepackage{wrapfig} % 本文が図を回り込むように配置
\usepackage{makeidx} %索引
\usepackage{hyperref} %ハイパーリンク
% \hypersetup{colorlinks=true,linkcolor=blue,citecolor=green}

% 和文フォント
% \usepackage[no-math,ipaex]{luatexja-preset} % IPAexフォント
\usepackage[no-math]{luatexja-fontspec} %Source Han フォント
	\setmainjfont{SourceHanSerifJP}
	\setsansjfont{SourceHanSansJP}
\ltjsetparameter{jacharrange={-2}} % 非ASCII文字がすべて和文と解釈されるのを防ぐ

% 欧文フォント
\setmainfont[Ligatures=TeX]{SourceSerifPro}
% \setmainfont[Ligatures=TeX]{XITS}
\setsansfont[Ligatures=TeX]{SourceSansPro}
\setmonofont[Ligatures=TeX]{SourceCodePro}

% 数式フォント
\usepackage{unicode-math}
\unimathsetup{math-style=ISO,bold-style=ISO}
\setmathfont{LatinModernMath}

% 索引を作成
\makeindex

% jsbook用の設定
\setlength{\textwidth}{\fullwidth} %本文領域を最大化 (余白を小さくする)
\setlength{\evensidemargin}{\oddsidemargin}

% 図表のキャプションの設定
\usepackage{caption}
\captionsetup[figure]{format=plain, labelformat=simple, labelsep=colon, font=small}
\captionsetup[table]{format=plain, labelformat=simple, labelsep=colon, font=small}

% 図表に通し番号を振る
\usepackage{remreset}
\makeatletter
	\@removefromreset{figure}{chapter}
	\def\thefigure{\arabic{figure}}
	\@removefromreset{table}{chapter}
	\def\thetable{\arabic{table}}
	\@removefromreset{equation}{chapter}
	\def\theequation{\arabic{equation}}
\makeatother

% MusiXTeX用設定
\usepackage{musixtex}
\input{musixadd.tex}
\nobarnumbers %小節番号なし
\smallmusicsize %楽譜のサイズを小さく
\newcounter{mycounter} % カウンタの宣言
\setcounter{mycounter}{0} % カウンタの初期化
\newcommand{\useMycounter}[1][]{\refstepcounter{mycounter}{#1}譜例{\themycounter}: }
\newcommand{\musicbegin}{\vspace*{5truemm}\begin{music}}
\newcommand{\musicend}[2]{\end{music}\nopagebreak\begin{center}{\small\useMycounter[\label{#1}]{#2}}\end{center}}

% 小節番号表示用
\newcommand{\shosetu}[1]{%
\startbarno=#1%
\systemnumbers%
\def\writebarno{\llap{\the\barno\barnoadd}}%
\def\raisebarno{2\internote}%
\def\shiftbarno{1.3\Interligne}%
}

% 構造番号
\newcommand{\ind}[2]{$\text{#1}_{\text{#2}}$}

% 文書データ
\title{ブラームス: クラリネットソナタ}
\author{H.~S.}
\date{2017.06.19-}

\begin{document}

\maketitle

\tableofcontents

% 
\chapter*{はじめに}
\addcontentsline{toc}{chapter}{はじめに}
\markboth{はじめに}{}



\chapter{作曲に関する経緯}

\section{背景と作曲過程}

Brahmsが晩年の室内楽分野における傑作, クラリネット三重奏曲とクラリネット五重奏曲を作曲したのは1891年で,
これはマイニンゲン宮廷管弦楽団の主席クラリネット奏者Richard Mühlfeldの演奏に触発されたものであることは広く知られている.
その後Brahmsはピアノのための小品 (作品116から119) に重点を移しているが,
クラリネットとピアノのための作品というアイディアは持っていたことがFerdinand Schumannの回想で伝わっている\cite{library}.
このアイディアが実現するのは1894年, Brahmsはイシュル滞在中の7月に2曲のクラリネットソナタを平行して作曲している\cite{compos}.
ただしイシュルに行く前にこの曲のスケッチを残している\cite{library}.

この時期のBrahmsの周辺は栄光と親しい友人との死別で彩られている.
1892年にはElisabeth von Herzogenbergが1月7日に亡くなっているし, 1893年2月26日に女性歌手だったHermine Spiesが急逝している.
1894年の2月6日にはTheodor Billroth, 2月12日にはHans von Bülowと立て続けに二人の親友が亡くなり, さらに4月13日にはPhilipp Spittaが亡くなっている.
一方でBrahmsの名声は頂点にあり, 1893年の60歳の誕生日に際して大規模な祝典が企画されたが,
Brahmsはこれを辞退してWidmannやHanslick夫妻とイタリア旅行へ出かけている\footnote{回想録集第3間p.140-145に記述あり.}.
また, 若い頃に渇望していた地位であるハンブルク市のフィルハーモニー協会音楽監督への就任依頼が1894年4月に届く.
これらの体験は「枯淡の境地」と呼ばれる後期ピアノ曲集に反映されていると考えられている.
それに対して, 作品120はそれを突き抜けたシンプルさ, 明るさが特徴的である.
どちらのソナタも, MozartやMendelssohnを思わせる明快さ, そして洗練された作曲技法を備えている.
主題展開に関しても, 以前のような変奏の技法を駆使するよりも,
比較的原型を留めたままの形でごくわずかな変更で絶妙なニュアンスを出す方向へと舵を切っている.
これらの傾向は1996年の4つの厳粛な歌 op.121でさらに推し進められることになる.


\section{初演と出版}

初演に先立つ1894年9月23日に, ベルヒテスガーデンにてマイニンゲン公爵らの立ち合いのもと, BrahmsはMühlfeldとこの2曲のソナタを試演している\cite{compos}.
続いて11月にはClara Schumann, Joseph Joachimの前でも演奏している\cite{henle}.
公開初演はウィーンの楽友協会にて, 第2番が1895年1月8日, 第1番が1月11日に行われた\footnote{そのリハーサルに関する記述が回想録集第2巻p.123にある.}\cite{library}.
恒例の演奏旅行がその後に続き, 3月までにベルリン, ライプツィヒ, フランクフルト, メルゼブルク, マイニンゲンで演奏している\cite{compos}.

% 評価

出版は1895年6月, Simrock社から. 同時にBrahms自身によるヴィオラ譜, またヴァイオリンとピアノのための編曲版も出版された\cite{imslp}.
Brahmsは後者の版についてはピアノパートにも手を入れている.


% \input{./tex/3-performances.tex}


\chapter{クラリネットソナタ第1番}

\section{全体の構造}

\section{第1楽章}

\section{第2楽章}

\section{第3楽章}

\section{第4楽章}



\chapter{クラリネットソナタ第2番}

% \section{全体の構造}

\section{第1楽章: Allegro amabile}

\begin{table}[htbp]
	\centering
	\begin{tabular}{ccc|ccc|ccc|c}
		\multicolumn{3}{c|}{提示部E} & \multicolumn{3}{c|}{展開部D} &
			\multicolumn{3}{c|}{再現部R} & コーダC \\ \hline
		\multicolumn{3}{c|}{1--55} & \multicolumn{3}{c|}{56--102} &
			\multicolumn{3}{c|}{103--149} & 150--173 \\
		\ind{E}{1} & \ind{E}{2} & \ind{E}{c} & \ind{D}{1} & \ind{D}{2} & \ind{D}{3} &
			\ind{R}{1} & \ind{R}{2} & \ind{R}{c} & \\ \hline
		1- & 22- & 40- & 56- & 73- & 88- & 103- & 120- & 138- & \\
		Es & B & B & Es--g & G & B & Es & Es & Es & E--Es
	\end{tabular}
	\caption{第1楽章の構成}
	\label{structure of mov1}
\end{table}


典型的なソナタ形式の楽章だが, Brahmsのどのソナタ楽章と比較しても形式的明瞭さが顕著である. 変ホ長調, 4分の4拍子.

\musicbegin
	\def\nbinstruments{2}%   % パート数 2
	\setstaffs{1}{2}%        % 下から1番目は2段
	\setclef{1}{6000}%       % 下から1番目はへ音記号
	\generalsignature{-3}%    % 調号は正の値のときシャープの数
	\generalmeter{\meterC}%  % 拍子
	\setname{2}{Cl\hspace{8truemm}}%
	\setname{1}{Pf\hspace{8truemm}}%
	\startextract%
		%(1)
		\notes\ibu{0}{G}{+2}\zhl{E}\qb{0}{E}\zqb{0}{I}\pt{H}\ql{I}\cu{**}\zh{I}\hu{E}|%
			\cu{**}\qb{0}{N}\tbl{0}\qb{0}{e}\ibl{1}{e}{-3}\qb{1}{gbe}\tbl{1}\qb{1}{N}&%
			\isluru{2}{l}\qlp{l**}\cl{k}\ibl{2}{j}{-1}\qb{2}{mlg}\tbl{2}\tslur{2}{j}\qb{2}{j}\enotes
		\bar
		%(2)
		\notes|%
			&%
			\ibsluru{2}{h}\hu{h***}\tbsluru{2}{d}\qu{d*}\ds\cu{c}\enotes
		\bar
		%(3)
		\notes|%
			&%
			\qup{e**}\cu{b}\ql{i*}\ds\cu{d}\enotes
	\endextract % 頭にzをつけると最後に小節線を表示しない
\musicend{2-1-1}{第2番第1楽章冒頭. クラリネットはB管だがここでは実音表記. }


序をおかずクラリネットによる第1主題の提示から始まる (譜例\ref{2-1-1}).

\musicbegin
	\shosetu{22}
	\def\nbinstruments{1}%   % パート数 1
	\setstaffs{1}{1}%        % 下から1番目は2段
	\setclef{1}{0000}%       % 下から1番目はへ音記号
	\generalsignature{-3}%    % 調号は正の値のときシャープの数
	% \generalmeter{\meterC}%  % 拍子
	\startextract%
		\notes\islurd{0}{M}\qu{M*}\tslur{0}{f}\qu{f*}\ds\ibu{0}{f}{+1}\islurd{0}{f}\qb{0}{ff}\tbu{0}\tslur{0}{g}\qb{0}{g}\enotes
		\bar
		\notes\islurd{0}{g}\qu{g*}\tslur{0}{c}\qu{c*}\ds\ibu{0}{f}{0}\islurd{0}{c}\qb{0}{cf}\tbu{0}\tslur{0}{c}\qb{0}{c}\enotes
		\bar
		\notes\islurd{0}{d}\qup{d**}\tslur{0}{b}\cu{b}\islurd{0}{b}\qup{b**}\na{a}\tslur{0}{a}\cu{a}\enotes
		\bar
		\Notes\na{a}\islurd{0}{a}\qu{a}\qu{b}\tslur{0}{c}\qu{c}\qp\enotes
	\endextract % 頭にzをつけると最後に小節線を表示しない
\musicend{2-1-22}{第2番第1楽章第22小節から.}

\musicbegin
	\shosetu{40}
	\def\nbinstruments{1}%   % パート数 1
	\setstaffs{1}{1}%        % 下から1番目は2段
	\setclef{1}{0000}%       % 下から1番目はへ音記号
	\generalsignature{-3}%    % 調号は正の値のときシャープの数
	% \generalmeter{\meterC}%  % 拍子
	\startextract%
		\Notes\qp\isluru{0}{p}\ql{p}\na{o}\ql{o}\tslur{0}{l}\ql{l}\enotes
		\bar
		\notes\ibl{0}{m}{0}\na{o}\isluru{0}{o}\qb{0}{onk}\tbl{0}\tslur{0}{n}\qb{0}{n}\isluru{0}{m}\ql{m*}\tslur{0}{j}\ql{j*}\enotes
		\bar
		\notes\ds\ibl{0}{k}{-2}\isluru{0}{k}\qb{0}{kl}\tbl{0}\tslur{0}{i}\qb{0}{i}\enotes
		\nnotes\ibbl{0}{j}{-1}\isluru{0}{k}\qb{0}{kj}\fl{g}\qb{0}{g}\tbl{0}\tslur{0}{j}\qb{0}{j}%
			\ibbl{0}{h}{+2}\isluru{0}{f}\qb{0}{fik}\tbl{0}\tslur{0}{m}\qb{0}{m}\enotes
		\bar
		\Notes\ql{p}\enotes
	\zendextract % 頭にzをつけると最後に小節線を表示しない
\musicend{2-1-40}{第2番第1楽章第40小節から.}

\section{第2楽章}

\section{第3楽章}



\begin{thebibliography}{99}
	\bibitem{library} 「作曲家別名曲解説ライブラリー ブラームス」音楽之友社 (1993)
	\bibitem{compos} 西原 稔 「作曲家 人と作品シリーズ ブラームス」音楽之友社 (2006)
	\bibitem{kaisouroku} 「ブラームス回想録集」全三巻, 音楽之友社 (2004)
	\bibitem{henle} 楽譜 G.~Henle Verlag版
	\bibitem{imslp} IMSLP (\href{http://imslp.org/wiki/Clarinet_Sonata_No.1,_Op.120_No.1_(Brahms,_Johannes)}{第1番},
		\href{http://imslp.org/wiki/Clarinet_Sonata_No.2,_Op.120_No.2_(Brahms,_Johannes)}{第2番})
	\bibitem{cd} ブラームス : ヴィオラ・ソナタ集 (今井信子/ヴィグノールス) (Chandos, 1987) CHAN8550
\end{thebibliography}


\printindex

\end{document}
